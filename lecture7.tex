% Copyright (C) 2014 by Massimo Lauria
% 
% Created   : "2014-01-07, Tuesday 17:01 (CET) Massimo Lauria"
% Time-stamp: "2014-03-03, 16:19 (CET) Massimo Lauria"
% Encoding  : UTF-8

% ---------------------------- USER DATA ------------------------------
\def\DataTitle{7. Sum of squares lower bounds for 3-SAT and 3-XOR\\ Part 1/2.}
\def\DataTitleShort{
Linear programming
}
\def\DataDate{17 February, 2014}
\def\DataDocname{Lecture 7 --- \DataDate}
\def\DataLecturer{Massimo Lauria}
\def\DataScribe{Ilario Bonacina}
\def\DataKeywords{
% should I put something here?
  }

\def\DataAbstract{%
  \stressterm{Disclaimer: this lecture note has not yet been reviewed
    by the main lecturer. It is released as it is for the convenience
    of the students.}

This lecture is about Positivstellensatz Calculus and a lower bound for the degree in that proof system of the so called random $k$-XOR formulas. We define the Positivstellensatz Calculus, the Binomial Calculus and prove a lower bound for the degree of random $k$-XOR formulas in Binomial Calculus. The proof of why this result leads to a lower bound for Positivstellensatz is in the next lecture.
  }


% ---------------------------- PREAMBLE -------------------------------
\documentclass[a4paper,twoside,justified]{tufte-handout}
\usepackage{soscourse} % this is a non standard package
\begin{document} 
% --------------------------- DOCUMENT --------------------------------

\section{Positivstellesatz Calculus ($\mathsf{PC}_>$)}

Let us consider the ordered field of the reals $\RR$, a finite set of variables $X$ and a finite set $P$ of polynomial equations in the ring $\RR[X]$. 

A \emph{derivation of  $p\geq 0$ from $P$ in $\mathsf{PC}_>$} is a sequence of polynomial equations ending with $p'=0$ such that $p=p'+\sum_i h_i^2$, where $h_i$ are polynomials in $\RR[X]$.  
Each polynomial equation in the sequence is either from $P$ or is the result of an inference from polynomial equations appearing previously in the sequence according to the following inference rules:
\begin{equation}
\label{eq:positivstellesatz-inference}
    \frac{q=0}{xq=0}\ x\in X,\qquad \frac{q=0\quad r=0}{\alpha q +\beta r=0}\ \alpha,\beta\in \RR.
\end{equation}
%
% Ilario: I stated the rules using equations just to have a visual reminder of the fact that all the things we can derive are actually to be intended equal to 0.
%
A \emph{refutation of $P$ in $\mathsf{PC}_>$} is a derivation of $-1\geq 0$ starting from $P$. 
The \emph{degree} of a derivation of $p\geq 0$ is the maximum degree of the intermediate polynomials appearing in the derivation of $p'$ and the maximum degree of the $h_i^2$'s.
 
 
 %
 % Ilario: I'm trying to avoid any unnecessary new notation for Binomial Calculus.
 %
The \emph{Binomial Calculus} (BC) is a particular case of the previous proof system. A \emph{BC derivation} of some binomial equation $p=0$\footnote{
We recall that a binomial is sum of two terms, each of them of the form $\alpha\prod x_i$ for some $\alpha\in \RR$ and for some subset of variables $x_i$ from $X$.
} from some set of binomial equations $Q$
is a $PC_<$ derivation of $p\geq 0$ from $Q$ where each intermediate polynomial equation is actually a binomial equation and the $\sum_ih_i^2$ part is $0$.

A \emph{refutation of $Q$} in BC is a derivation in BC of $\alpha -1=0$ for some $\alpha\in \RR$, $\alpha\neq 1$. The very same notion of degree of $\mathsf{PC}_>$ apply here.

%
% Ilario: I have not put the observation about $\alpha m=1$ and the degree because that is needed only in next lecture. If the scribe of next lecture have not written it I'm going to put it, here or there.
%

\section{Random $k$-XOR formulas}

Let $X=\{x_1,\ldots,x_n\}$ be a set of variables and $m,k\in \NN$. Sample uniformly at random $b\in \{0,1\}$ and $S\subset [n]$ of size $k$ and from them build the parity constraint $\sum_{i\in S}x_i \equiv b \pmod{2}$.
Repeat independently at random this process $m$ times to obtain a \emph{random $k$-XOR formula} on variables in $X$ with $m$ \emph{parity constraints}\footnote{A very similar process is used to build \emph{random} $k$-SAT formulas: pick uniformly at random a set $S\subset [n]$ of size $k$ and a random mapping $b:S\rightarrow \{0,1\}$. From those build the clause $\bigvee_{i\in S} x_i^{b(i)}$, where $x^1:=x$ and $x^0:=\neg x$. Repeat independently at random this process $m$ times and take the conjunction of the clauses you get.}. 

We can associate to a random $k$-XOR formula a set of polynomial equations such that the formula has a boolean solution iff the set of polynomial equations has a solution. 
The encoding of a single parity constraint $\sum_{i\in S}x_i \equiv b \pmod{2}$ as a set of polynomial equations in $\RR[X]$ is the following:
\begin{equation}\label{eq:kXORx}
\left\{\prod_{i\in S}(1-2x_i) = (-1)^b \right\}_S \cup \{x_i^2=x_i\}_{i \in X}.
\end{equation} 

In what follow we will use another encoding using a different set of variables $Y=\{y_1,\ldots,y_n\}$.
 In this case the parity constraint $\sum_{i\in S}x_i  \equiv b \pmod{2}$ has a solution iff the following set of polynomial equations in $\RR[Y]$ has a zero:
\begin{equation}\label{eq:kXORy}
\{\prod_{i\in S}y_i = (-1)^b \}\cup \{y_i^2=1\}_{i\in S}.
\end{equation} 

Obviously there is a linear transformation from $\RR[X]$ to $\RR[Y]$ mapping the first set into the second: $x_i\mapsto (y_i-1)/2$.

Notice that the degree of $\mathsf{PC}_>$ refutations of an unsatisfiable set of parity constraints does not depend on whether the encoding as polynomial equations is the one in (\ref{eq:kXORx}) or (\ref{eq:kXORy}). 
As already observed there is a linear mapping from one set to the other so we can apply that mapping to a $\mathsf{PC}_>$ refutation over $\RR[Y]$ to obtain a valid $\mathsf{PC}_>$ refutation over $\RR[X]$ (and vice versa) both having the same degree.

\begin{theorem}\label{thm:kXOR}
For each $k\geq 3$ and $\delta>0$ there exists $\alpha$, such that a random $k$-XOR formula $\phi$ in $n$ variables and $\Delta n$ clauses, where $\Delta \geq (1+\ln 2)\frac{1}{2\delta^2}$, with high probability has the following properties:
\begin{enumerate}
\item At most $\left( \frac{1}{2}+\delta\right)\Delta n$ parity constraints of $\phi$ can be simultaneously satisfied,
\item Any $\mathsf{PC}_>$ refutation of $\phi$ requires degree $\alpha n$.
\end{enumerate}
\end{theorem}

\begin{proof}[Proof of part 1 of the Theorem.]
Given $\phi$ we proceed by applying Chernoff Bound and then union bound. Lets fix an assignment $x\in \{0,1\}^n$ and let $C_i(x)$ be the random variable that is $1$ if $x$ satisfy the $i$-th parity constraint in $\phi$ and $0$ otherwise. Hence $\sum_i C_i(x)$ is the number of linear constraints of $\phi$ satisfied by $x$. Then $\mathbb{E}[C_i(x)]=\frac{1}{2}$ and by linearity $
\mathbb{E}[\sum_i C_i(x)]=\frac{1}{2}\Delta n$.
Hence, by Chernoff Bound\footnote{
We use the (standard) following form of Chernoff Bound: let $X_1,\ldots, X_m$ be independent 0-1 random variables and $X=\sum_{i\in [m]} X_i$ then for every $\lambda >0$ 
\begin{equation*}
\mathbb{P}\left[X\geq \mathbb{E}[X]+\lambda\right]\leq e^{-\frac{2\lambda^2}{m}}.
\end{equation*}

}, for any $\delta>0$,
\begin{equation*}
\mathbb{P}\left[\sum_i C_i(x)\geq (\frac{1}{2}+\delta)\Delta n\right]\leq e^{-2\delta^2\Delta n}.
\end{equation*}
Hence by union bound 
\begin{equation*}
\mathbb{P}\left[\exists x\in \{0,1\}^n\ \left(\sum_i C_i(x)\geq (\frac{1}{2}+\delta)\Delta n\right)\right]\leq 2^n \cdot e^{-2\delta^2\Delta n}\leq e^{-n}.
\end{equation*}
The last inequality comes from the assumption that $\Delta\geq (1+\ln 2)\frac{1}{2\delta^2}$.
\end{proof}
Before going deep into the proof of part 2. of Theorem \ref{thm:kXOR} we just state and prove an interesting corollary.

\begin{corollary}\label{cor:randomCNF}
For each $k\geq 3$ and $\delta>0$, 
there exists an $\alpha$, 
such that with high probability
for a random $k$-SAT formula $\phi$ with $\Delta n$ clauses and $\Delta\geq (1+\ln 2)\frac{1}{2\delta^2}$:
\begin{enumerate}
\item At most $\left(\frac{2^k-1}{2^k}+\delta\right)\Delta n$ clauses of $\phi$ can be satisfied at the same time and
\item Any $\mathsf{PC}_>$ refutation of $\phi$ requires degree at least $\alpha n$.
\end{enumerate}
\end{corollary}

\begin{proof}
The proof of point 1 is exactly the same of the analogous point of Theorem \ref{thm:kXOR}. 
The only difference is that the expected value of the random variable representing the number of clauses satisfied changes to $\frac{2^k-1}{2^k}\Delta n$. 
The rest of the calculations are exactly the same.

Regarding the second point we just observe that from random $k$-XOR we can derive in degree $(k+1)$ random $k$-SAT.

For each parity constraint $\sum_{i\in S}x_i\equiv b \pmod{2}$ in random $k$-XOR we choose uniformly at random one of the clauses derivable from that constraint\footnote{
For example consider the parity constraint $x_1+x_2+x_3\equiv 0 \pmod{2}$, that has polynomial encoding as $(1-2x_1)(1-2x_2)(1-2x_3)=1$, that is the same of 
$x_1+x_2+x_3-2(x_1x_2+x_1x_3+x_2x_3)+4x_1x_2x_3=0$. 
From this, multiplying by $x_1$, $x_2$ and $x_3$ we can derive (in degree 4)
\begin{equation*}
\begin{split}
x_1-x_1x_2-x_1x_3 +2x_1x_2x_3=0,\\
x_2 -x_1x_2-x_2x_3+2x_1x_2x_3=0,\\
x_3 -x_3x_2-x_1x_3+2x_1x_2x_3=0.
\end{split}
\end{equation*}
Summing all those and subtracting the initial one we get $x_1x_2x_3=0$, that is the encoding of $\neg x_1\vee \neg x_2 \vee \neg x_3$.
}.
That is half of the possible $k$ clauses in the variables $\{x_i\}_{i\in S}$ are cut away and the other half is derivable in degree $k+1$ from $\prod_{i\in S}(1-2x_i)=(-1)^b$. 
The $k$-SAT formulas we obtain in this way have a distribution indistinguishable from that of random $k$-SAT. Hence for sufficiently large $n$ it is not possible to derive in small degree random $k$-SAT, otherwise random $k$-XOR would have small degree refutations too but this is excluded by Theorem \ref{thm:kXOR}.
\end{proof}

The previous Theorem and the Corollary show in particular that after $\alpha n$ steps in the Lasserre hierarchy the integrality gap 
is $1/2 +\delta$ for Max $k$-XOR and $\frac{2^k-1}{2^k}+\delta$ for Max $k$-SAT. 
This means that for both of those problems the integrality gap cannot be much better than $1/2$ or $\frac{2^k-1}{2^k}$ respectively.

\section{Proof of Theorem \ref{thm:kXOR} (Part 2)}

As the proof is quite long, we recap briefly its high level structure:
\begin{itemize}
\item Observe that to prove a degree lower bound for $k$-XOR formulas, it is irrelevant if we choose the encoding in (\ref{eq:kXORx}) or (\ref{eq:kXORy}). 
So, to make our life easier, we choose the encoding in (\ref{eq:kXORy}).
\item Up to a constant factor of $2$, it is the same to prove a degree lower bound for the binomial encoding of a random $k$-XOR over $\RR[Y]$ in BC or for the other encoding in $\mathsf{PC}_>$. See next Lecture.
\item Actually prove a degree lower bound in BC for the encoding (\ref{eq:kXORy}) of a random $k$-XOR over $\RR[Y]$.
\end{itemize}

The remaining part of this lecture is devoted to proving the last
point above. We premise a Lemma about the structure of random $k$-XOR
formulas. The proof is omitted but follows immediately from
Proposition 22 in (Schoenebeck, 2008)\cite{schoenebeck2008linear}.
 

\begin{lemma}\label{lem:kXORsat}
Given constants $k\geq 3$, $\Delta>0$ and $\gamma\in (0,k/2)$, there exists a $\beta$, such that for a random $k$-XOR formula with $n$ variables and $\Delta n$ parity constraints with high probability the following hold
\begin{enumerate}
\item for each $\phi'\subseteq \phi$ if $|\phi'|\leq \beta n$ then $\phi'$ is satisfiable,
\item for each $\phi'\subseteq \phi$ if $|\phi'|\leq \frac{2}{3}\beta n$ then there are at least $\gamma |\phi'|$ variables appearing once in $\phi'$.
\end{enumerate}

\end{lemma}

\begin{theorem}
Given constants $k\geq 3$, $\Delta>0$ and $\gamma \in (0,k/2)$, there exists $\alpha$, such that 
with high probability, for a random $k$-XOR formula $\phi$ in $n$ variables and $\Delta n$ constraints,
every BC refutation of $\phi$ over $\RR[Y]$\footnote{
That is every BC refutation of the encoding (\ref{eq:kXORy}) of $\phi$ over $\RR[Y]$.
} 
require degree at least $\alpha n$.
\end{theorem}

\begin{proof}
Let $B$ the set of all binomial equations we can derive from $\phi$ in Binomial Calculus. 
We define a measure $\mu:B\rightarrow \RR$ as follows\footnote{
$\phi'\models p$ means that the set of parity constraints $\phi'$ \emph{imply} the equation $p$, ie the satisfying assignmnets of $\phi'$ are also satisfying assignments of $p$.

Similarly, for an assignment $\beta$ and a formula $\phi$, $\beta \models \phi$ means that the assignment $\beta$ satisfies
all constraints in $\phi$.
}
:
\begin{equation}\label{eq:mu-kXOR}
\mu(p):=\min\{|\phi'|\ :\ \phi'\subseteq \phi\ \wedge\ \phi'\models p\}.
\end{equation}
Clearly for each binomial $b$ appearing in the encoding of $\phi$ we have that $\mu(b)= 1$ and $\mu$ is sub-additive wrt the inference rules in (\ref{eq:positivstellesatz-inference}). This is immediate from the definition of $\mu$ that if $\{p,q\}\models r$ then $\mu(r)\leq \mu(p)+\mu(q)$. 

Let us now consider a refutation $\pi$ of $\phi$ in BC, say ending with $\eta=1$ for some $\eta\in \RR$, $\eta\neq 1$. By Lemma \ref{lem:kXORsat} we have that $\mu(\eta=1)>\beta n$.

By the sub-additivity of $\mu$, we have that there exists some medium complexity binomial equation in $\pi$. 
More precisely there exists a binomial equation $q$ in $\pi$ such that
\begin{equation*}
\frac{1}{3}\beta n< \mu(q) \le \frac{2}{3}\beta n.
\end{equation*}

Just take as $q$ the first binomial appearing in $\pi$ such that $\mu(q)>\frac{1}{3}\beta n$. $q$ must have been inferred by previous binomials. 
By the fact that $q$ is the \emph{first} binomial in $\pi$ having big $\mu$ and by sub-additivity of $\mu$ we have also the other inequality
$\mu(q) \leq \frac{2}{3} \beta n$.

We want now to prove that $q$ has large degree. Let $\phi'\subseteq \phi$ such that $\phi'\models q$. 
By the above inequality we have that $\frac{1}{3}\beta n < |\phi'| \leq \frac{2}{3}\beta n$, hence by Lemma \ref{lem:kXORsat} we have at least $\gamma |\phi'|$ single variables in $\phi'$. 
If we prove that those variables have to appear also in $q$ we are done: as $q$ is a binomial this means that $\deg(q)\geq \gamma |\phi'|/2\geq \frac{1}{6}\beta n$. 
Then the parameter $\alpha$ of the statement of the Theorem is just $\frac{1}{6}\beta$.

We now prove that each variable that appears once in $\phi'$ has to appear in $q$ too. 
Suppose by contradiction there is some variable $y_i$ appearing once in $\phi'$ and not appearing in $q$. 
This variable appears only in one parity constraint of $\phi'$, say $l$. Consider $\bar \phi=\phi'\setminus \{l\}$. 
By minimality of $\phi'$ there exists an assignment $\beta$ such that $\beta \models \bar \phi$ and $\beta(q)=0$\footnote{
Where we use the standard meaning of $0$=False and $1$=True.
}
. 
Then just take $\beta^*$ an assignment that disagree with $\beta$ only on the value given to $y_i$. 
This imply that $\beta^*(q)=\beta(q)=0$, as $y_i$ does not appear in $q$.
 But also that $\beta^*(l)=1-\beta(l)=1$, as flipping a single value in a parity constraint flip also the truth value of the constraint. 
 Hence $\beta^*\models \phi$ and $\beta^*(q)=0$ in contradiction with the fact that $\phi\models q$.
\end{proof}

% ------------------------- EPILOGUE ------------------------------
\bibliography{soscourse}
\bibliographystyle{alpha}

\end{document} 


%%% Local Variables:
%%% mode: latex
%%% TeX-master: t
%%% End:
