% Copyright (C) 2014 by Massimo Lauria
% 
% Created   : "2014-01-07, Tuesday 17:01 (CET) Massimo Lauria"
% Time-stamp: ""
% Encoding  : UTF-8

% ---------------------------- USER DATA ------------------------------
\def\DataTitle{2. Semidefinite programming and Relaxation}
\def\DataTitleShort{Semidefinite programming}
\def\DataDate{28 January, 2014}
\def\DataDocname{Lecture 2 --- \DataDate}
\def\DataLecturer{Massimo Lauria}
\def\DataScribe{Massimo Lauria}
\def\DataKeywords{semidefinite programming, positive semidefinite matrices}

\def\DataAbstract{%
  Semidefinite programs have been proved to be valuable tools in
  approximation algorithms and in combinatorial optimization, since
  semidefinite relaxations are usually stronger than linear one. In
  lecture we describe what is a semidefinite program and we
  introduce hierarchies of sdp relaxations.}

% ---------------------------- PREAMBLE -------------------------------
\documentclass[a4paper,twoside,justified]{tufte-handout}
\usepackage{soscourse} % this is a non standard package
\begin{document} 
% --------------------------- DOCUMENT --------------------------------

\section{Positive semidefinite matrices}

A fundamental concept in semidefinite programming is the one of
\introduceterm{positive semidefinite matrices}. 
\begin{definition}
  A square matrix $A$ in $ \RR^{n\times n} $ is called positive
  semidefinite if 
  \begin{itemize}
    \item $A$ is symmetric\footnote{the condition of symmetry is often
      forgotten, but it is required.}, \ie, $ A=A^{T} $;
    \item for every $ x\in R^{n} $ it holds that $ x^{T} A x \geq 0 $.
  \end{itemize}
  A matrix $ A $ is \introduceterm{positive definite} if it is
  positive semidefinite and is also non-singular.
\end{definition}
We denote the set of positive semidefinite matrices in $\RR^{n\times
  n}$ as $\positivesemidefinite$.

\begin{fact}
  $ \positivesemidefinite $ is closed under positive
  combinations (\ie, is a cone): for a sequence $ M_{1}, M_{2}, \ldots, M_{\ell} $ of positive
  semidefinite matrices, the matrix 
\begin{equation}
  \alpha_{1} M_{1} + \alpha_{2} M_{2} + \cdots + \alpha_{\ell} M_{\ell}
\end{equation}
is positive semidefinite for $ \alpha_{1}\geq 0, \alpha_{2}\geq 0,
\ldots , \alpha_{\ell} \geq 0$.
\end{fact}

\section{Matrix decompositions}

Positive semidefinite matrices have a lot of nice properties, and in
particular they have a set of useful decomposition. If a matrix $ A
\in \positivesemidefinite[n] $ then there are decompositions

\begin{align}
  A = U^{T}U & & \text{Cholesky decomposition}\\
  A = L D L^{T} & & \text{LDL decomposition}\\
  A = Q \Lambda Q^{T} & & \text{Spectral decomposition}
\end{align}

with $ U,L,D,Q,\Lambda $ are real matrices in $ \RR^{n\times n} $

In \introduceterm{spectral decomposition}, matrix $ Q $ is an
\introduceterm{orthogonal matrix} which columns are unitary
eigenvectors of $ A $, \ie $ QQ^{T} = I$, and $ \Lambda $ is the
diagonal matrix containing the corresponding eigenvectors.

\begin{figure*}
\begin{equation}
A=
% Q matrix
\begin{bmatrix}
    q_{11}  & q_{12} & \ldots & q_{1n}\\
    q_{21}  & q_{22} & \ldots & q_{2n}\\
    & \ddots & &  \\
    q_{n1}  & q_{n2} & \ldots & q_{nn}
\end{bmatrix}%
\cdot
% lambda matrix
\begin{bmatrix}
    \lambda_{1}  & 0 & \ldots & 0\\
    0 & \lambda_{2} & \ldots & 0\\
    & \ddots & &  \\
    0  & 0 & \ldots & \lambda_{n}
\end{bmatrix}
\cdot
\begin{bmatrix}
    q_{11}  & q_{21} & \ldots & q_{n1}\\
    q_{12}  & q_{22} & \ldots & q_{n2}\\
    & \ddots & &  \\
    q_{1n}  & q_{2n} & \ldots & q_{nn}
\end{bmatrix}
\end{equation}
\end{figure*}

\begin{equation}
  \begin{bmatrix}
    q_{i1}  & q_{i2} & \ldots & q_{in}\\
   \end{bmatrix}
  \cdot
  \begin{bmatrix}
    q_{j1}  \\
    q_{j2}  \\
    \vdots \\
    q_{jn} 
  \end{bmatrix} = \begin{cases}
    0 & i\neq j\\
    1 & i = j
  \end{cases}
\end{equation}


\begin{equation}
  A \cdot
  \begin{bmatrix}
    q_{i1}  \\
    q_{i2}  \\
    \vdots \\
    q_{in} 
  \end{bmatrix} =
  \lambda_{i} \cdot
  \begin{bmatrix}
    q_{i1}  \\
    q_{i2}  \\
    \vdots \\
    q_{in} 
  \end{bmatrix}
\end{equation}

% ------------------------- EPILOGUE ------------------------------
\bibliography{soscourse}
\bibliographystyle{alpha}

\end{document} 


%%% Local Variables:
%%% mode: latex
%%% TeX-master: t
%%% End:
