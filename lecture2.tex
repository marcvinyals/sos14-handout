% Copyright (C) 2014 by Massimo Lauria
% 
% Created   : "2014-01-07, Tuesday 17:01 (CET) Massimo Lauria"
% Time-stamp: ""
% Encoding  : UTF-8

% ---------------------------- USER DATA ------------------------------
\def\DataTitle{2. Semidefinite programming and Relaxation}
\def\DataTitleShort{Semidefinite programming}
\def\DataDate{28 January, 2014}
\def\DataDocname{Lecture 2 --- \DataDate}
\def\DataLecturer{Massimo Lauria}
\def\DataScribe{Massimo Lauria}
\def\DataKeywords{semidefinite programming, positive semidefinite matrices}

\def\DataAbstract{%
  Semidefinite programs have been proved to be valuable tools in
  approximation algorithms and in combinatorial optimization, since
  semidefinite relaxations are usually stronger than linear one. In
  lecture we describe what is a semidefinite program and we
  introduce hierarchies of sdp relaxations.}

% ---------------------------- PREAMBLE -------------------------------
\documentclass[a4paper,twoside,justified]{tufte-handout}
\usepackage{soscourse} % this is a non standard package
\begin{document} 
% --------------------------- DOCUMENT --------------------------------

\section{Positive semidefinite matrices}

A fundamental concept in semidefinite programming is the one of
\introduceterm{positive semidefinite matrices}. 
\begin{definition}
  A square matrix $A$ in $ \RR^{n\times n} $ is called positive
  semidefinite if 
  \begin{itemize}
    \item $A$ is symmetric\footnote{the condition of symmetry is often
      forgotten, but it is required.}, \ie, $ A=A^{T} $;
    \item for every $ x\in R^{n} $ it holds that $ x^{T} A x \geq 0 $.
  \end{itemize}
  A matrix $ A $ is \introduceterm{positive definite} if it is
  positive semidefinite and is also non-singular.
\end{definition}
We denote the set of positive semidefinite matrices in $\RR^{n\times
  n}$ as $\positivesemidefinite$.

\begin{fact}
  $ \positivesemidefinite $ is closed under positive
  combinations. \Ie, let $ \{M_{i}\}^{\ell}_{i=1} $ a set of positive
  semidefinite matrices and $ \alpha \in \RR^{\ell} $ with $ \alpha
  \geq 0 $ then $ \sum^{\el}_{i} \alpha_{i} M_{i}$ is positive
  semidefinite.
\end{fact}

% ------------------------- EPILOGUE ------------------------------
\bibliography{soscourse}
\bibliographystyle{alpha}

\end{document} 


%%% Local Variables:
%%% mode: latex
%%% TeX-master: t
%%% End:
