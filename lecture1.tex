% Copyright (C) 2014 by Massimo Lauria
% 
% Created   : "2014-01-07, Tuesday 17:01 (CET) Massimo Lauria"
% Time-stamp: "2014-01-09, 11:16 (CET) Massimo Lauria"
% Encoding  : UTF-8

% ---------------------------- USER DATA ------------------------------
\def\DataTitle{1. Linear relaxations of integer programs.}
\def\DataTitleShort{Linear programming}
\def\DataDate{27 January, 2014}
\def\DataDocname{Lecture 1 --- \DataDate}
\def\DataLecturer{Massimo Lauria}
\def\DataScribe{Massimo Lauria}
\def\DataKeywords{integer programming, linear programming,
  \Lovasz-\Schrijver, Sherali-Adams, }

\def\DataAbstract{%
  We can express combinatorial problems using integer programs, but
  since we can't solve them, we consider relaxed linear programs and
  we round fractional solutions to integer. We consider hierarchies of
  linear programs and discuss the quality of the corresponding
  solutions.}


% ---------------------------- PREAMBLE -------------------------------
\documentclass[a4paper,twoside,justified]{tufte-handout}
\usepackage{soscourse} % this is a non standard package
\begin{document} 
\maketitle
\thispagestyle{fancy}
\begin{marginfigure}
  \emph{Lecturer: \DataLecturer}
\end{marginfigure}

\begin{abstract}
  \emph{\DataAbstract}
\end{abstract}
% --------------------------- DOCUMENT --------------------------------

% Start to write here.

This lecture is a sort of scaled down demo of the rest of the
course. Here we are see that we can express decisions and optimization
problems by the means of \introduceterm{integer programs}. This translation in
even possible for $ \NP $-hard problems, thus there is not efficient
algorithm to solve integer programs unless $\PTIME=\NP$, which is
considered by many to be very unlikely\footnote{Every hardness result
  we will see during the course won't rely on any unproved assumption.}.

In any case there are no \stressterm{known} efficient algorithm to
solve integer programs, and a viable strategy is to
\introduceterm{relax} the integer program to something more
manageable: for example a liner program.

The most naive way to do that is to transform the integrality
constraints into fractional linear constraints, \eg $ x \in\{0,1\} $
into $ 0 \leq x \leq 1 $, and leave the other constraints as they
are\footnote{We can assume that all such constraints are affine,
  namely of the three forms
  \begin{equation*}
    \sum_{i}a_{i}x_{i} \leq b \quad   \sum_{i}a_{i}x_{i} \geq b \quad \sum_{i}a_{i}x_{i} = b
  \end{equation*}
  for $a_i$ and $ b $ in \RR}.

Once we relax the integer program we have a lot of new fractional
solution that are not allowed in the integer one. For example consider
the program that characterizes the maximum independent sets of graph $
K_{3} $, \ie, the triangle.

\begin{alignat}{2}
  \maximize x_{1} + x_{2} + x_{3} \notag\\
  \subjectto   x_{1} + x_{2} \leq 1 \notag\\
             & x_{2} + x_{3} \leq 1 \label{eq:triangle-ind-set}\\
             & x_{1} + x_{3} \leq 1 \notag\\
             & \varboolean{x_{1}},\ \varboolean{x_{2}},\ \varboolean{x_{3}}.\notag
\end{alignat}

Its \introduceterm{integer optimum} is 1 since no two variables can be
both 1, but if we relax the constraints and allow $ \varbounded{x_{i}}
$, then the linear program has a \introduceterm{fractional optimum}
of $ \frac{3}{2} $, by setting all variables to $\frac{1}{2}$.

Most of the course will study systematic techniques to improve the
relaxation, adding variables and inequalities in order to
\stressterm{cut away} feasible fractional solutions without changing
the set of integer solutions. The quality and the complexity of such
techniques is controlled by a parameter called \introduceterm{rank}:
larger the rank, less fractional solution remain.

\section{Linear programming}

\section{Integer programs and Linear relaxations}


\section{Integrality gaps}


\section{Improving the linear relaxations}


\section{\Lovasz-\Schrijver\ hierarchy}

\subsection{Rank lower bounds}

\section{Sherali-Adams hierarchy}

\subsection{Rank lower bounds}


% ------------------------- EPILOGUE ------------------------------
\bibliography{theoryofcomputing,soscourse}
\bibliographystyle{alpha}

\end{document} 


%%% Local Variables:
%%% mode: latex
%%% TeX-master: t
%%% End:
